%&pdflatex
%
%  Clawpack 5.x (x <= 3) Paper
%
\documentclass[]{article}

\usepackage{graphicx}
\usepackage[colorinlistoftodos]{todonotes}

% Use utf-8 encoding for foreign characters
\usepackage[utf8]{inputenc}

% Multipart figures
% \usepackage{subcaption}

% More symbols
\usepackage{amsmath}
\usepackage{amssymb}
\usepackage{latexsym}

% Package for including code in the document
\usepackage{listings}
% \DeclareCaptionFont{white}{\color{white}}
% \DeclareCaptionFormat{listing}{%
%   \parbox{\textwidth}{\colorbox{gray}{\parbox{\textwidth}{#1#2#3}}\vskip-4pt}}
% \captionsetup[lstlisting]{format=listing,labelfont=white,textfont=white}
% \lstset{frame=lrb,xleftmargin=\fboxsep,xrightmargin=-\fboxsep,numbers=left,basicstyle=\ttfamily\small,language=python,columns=fixed,commentstyle=\em\color[rgb]{0.133,0.545,0.133}}

% This is now the recommended way for checking for PDFLaTeX:
\usepackage{ifpdf}

% Add line numbers
\usepackage[mathlines]{lineno}

% Need to use this package to handle the spacing issues in LaTeX after a command
\usepackage{xspace}

% Markup
\usepackage{color}
% \newcommand{\comment}[1]{{\color{blue}#1}}
% \newcommand{\alert}[1]{\textbf{\color{red} #1}}
\newcommand{\revised}[1]{{\color{red} #1}}

% URL and linking help with subtle colors and no boxes
\definecolor{darkgreen}{rgb}{0.1,0.5,0.1}
\definecolor{darkblue}{rgb}{0.2,0.2,1.0}
\usepackage[colorlinks=true,linkcolor=darkblue,citecolor=darkblue,
            filecolor=darkblue,urlcolor=darkgreen]{hyperref}


% Need this package to allow footnotes for more than 9 authors:
\usepackage{footmisc}

%=================
% Try using cleveref...

\usepackage[capitalize,nameinlink]{cleveref}

\crefformat{equation}{#2(#1)#3}
\crefrangeformat{equation}{(#3#1#4--#5#2#6)}
\crefmultiformat{equation}{#2(#1)#3}{ and #2(#1)#3}
{, #2(#1)#3}{, and #2(#1)#3}
\crefrangemultiformat{equation}{(#3#1#4--#5#2#6)}%
{ and (#3#1#4--#5#2#6)}{, (#3#1#4--#5#2#6)}{, and (#3#1#4--#5#2#6)}

%=================


\graphicspath{{./figures/}}


\setlength{\textwidth}{6.2in}
\setlength{\oddsidemargin}{0in}
\setlength{\evensidemargin}{0in}
\setlength{\textheight}{8.7in}
\setlength{\voffset}{-.7in}
\setlength{\headsep}{26pt}


% Useful commands
%!TEX root = paper.tex
%  Generic math macros
\renewcommand{\v}[1]{\boldsymbol{#1}}
\newcommand{\m}[1]{\text{\textsf#1}}

\newcommand{\pd}[2]{\ensuremath{\frac{\partial #1}{\partial #2}}} % partial
\newcommand{\dee}{\ensuremath{\mathrm{d}}}                  % d symbol
\newcommand{\diff}[2]{\ensuremath{\frac{\dee #1}{\dee #2}}} % Derivative d/dx
% \newcommand{\grad}{\ensuremath{\nabla}}                     % Gradient symbol
\newcommand{\gradient}{\ensuremath{\textsf{grad}}}          % Written gradient
\newcommand{\Div}{\ensuremath{\nabla \cdot}}                % Divergence
\newcommand{\divergence}{\ensuremath{\textsf{div}}}         % Written div
\newcommand{\del}{\ensuremath{\nabla}}                      % Same as gradient
\newcommand{\delsq}{\ensuremath{\nabla^2}}                  % Laplacian
\newcommand{\lap}{\ensuremath{\delsq}}                      % Laplacian
\newcommand{\dx}{\ensuremath{\Delta x}}                     % Delta x
\newcommand{\dy}{\ensuremath{\Delta y}}                     % Delta y
\newcommand{\dt}{\ensuremath{\Delta t}}                     % Delta t
\newcommand{\scinot}[2]{\ensuremath{#1\times10^{#2}}}       % Scientific note
\newcommand{\bigO}[1]{\ensuremath{\mathcal{O}(#1)}}         % Big O notation
\newcommand{\R}{\ensuremath{\mathbb{R}}}                    % Real field
\newcommand{\Z}{\ensuremath{\mathbb{Z}}}                    % Integer field
\newcommand{\half}{\ensuremath{\frac{1}{2}}}                % 1/2 fraction

%  Special formatting for codes
\newcommand{\geoclaw}{{\sc GeoClaw}\xspace}
\newcommand{\clawpack}{{\sc Clawpack}\xspace}
\newcommand{\amrclaw}{{\sc AMRClaw}\xspace}
\newcommand{\pyclaw}{{\sc PyClaw}\xspace}
\newcommand{\forestclaw}{Forestclaw\xspace}

%  Finite volume method symbols
\newcommand{\wave}{\ensuremath{\mathcal{W}}\xspace}             % Wave
\newcommand{\fwave}{\ensuremath{\mathcal{Z}}\xspace}            % F-Waves
\newcommand{\cell}{\ensuremath{\mathcal{C}}\xspace}             % FV grid cell
\newcommand{\apdq}{\ensuremath{\mathcal{A}^+ \Delta Q}\xspace}      % A+dq
\newcommand{\amdq}{\ensuremath{\mathcal{A}^- \Delta Q}\xspace}      % A-dq
\newcommand{\apmdq}{\ensuremath{\mathcal{A}^{\pm} \Delta Q}\xspace} % A+-dq

% B+-A+-dq symbols
\newcommand{\BpApdq}{\ensuremath{\mathcal{B}^{+} \mathcal{A}^{+} \Delta Q}\xspace}
\newcommand{\BpAmdq}{\ensuremath{\mathcal{B}^{+} \mathcal{A}^{-} \Delta Q}\xspace}
\newcommand{\BmApdq}{\ensuremath{\mathcal{B}^{-} \mathcal{A}^{+} \Delta Q}\xspace}
\newcommand{\BmAmdq}{\ensuremath{\mathcal{B}^{-} \mathcal{A}^{-} \Delta Q}\xspace}
\newcommand{\BpmApmdq}{\ensuremath{\mathcal{B}^{\pm} \mathcal{A}^{\pm} \Delta Q}\xspace}

\begin{document}

\ifpdf
\DeclareGraphicsExtensions{.pdf, .png, .jpg, .tif}
\else
\DeclareGraphicsExtensions{.png, .jpg, .tif, .eps}
\fi

\title{The Clawpack 5.X Software}

% Authors: Anyone who has....
%  - Made a nontrivial contribution to 5.0, 5.1, 5.2,
%  - Contributes at least a sentence to the paper,
%  - Has read the final draft and agreed to be an author.

\author{
        Kyle T. Mandli\thanks{
            Columbia University (\mbox{kyle.mandli@columbia.edu})} \and
        Aron J. Ahmadia\thanks{
            Continuum Analytics (\mbox{aahmadia@continuum.io})} \and
        Marsha Berger\thanks{
            New York University (\mbox{berger@cims.nyu.edu})} \and
        Donna Calhoun\thanks{
            Boise State University (\mbox{donnacalhoun@boisestate.edu})} \and
        % Matthew Emmett\thanks{
        %     Lawrence Berkeley National Laboratory (\mbox{memmett@gmail.com})} \and
        David George\thanks{
            USGS Cascades Volcano Observatory (\mbox{dgeorge@usgs.gov})} \and
        Yiannis Hadjimichael\thanks{
            King Abdullah University of Science and Technology, Box 4700, Thuwal, Saudi Arabia, 23955-6900 (\mbox{yiannis.hadjimichael@kaust.edu.sa})} \and
        David I. Ketcheson\thanks{
            King Abdullah University of Science and Technology, Box 4700, Thuwal, Saudi Arabia, 23955-6900 (\mbox{david.ketcheson@kaust.edu.sa})} \and
        Grady I. Lemoine\thanks{
            CD-Adapco, Bellevue, WA} \and
        Randall J. LeVeque\thanks{
            Dept.\ of Applied Mathematics, University of Washington (\mbox{rjl.@uw.edu})}
        }

\maketitle
\linenumbers
\begin{abstract}
    \clawpack is a software package designed to solve nonlinear
hyperbolic partial differential equations using high-resolution finite volume
methods based on Riemann solvers and limiters. The package includes a number
of variants aimed at different applications and user communities.
\clawpack has been actively developed as an open source project
for over 20 years.  The latest major release, \clawpack 5, introduces
a number of new features and changes to the code base and a new
development model based on GitHub and Git submodules.  
This article provides a summary of the most significant changes,
the rationale behind some of these changes, and a description of our current
development model.
\end{abstract}

%!TEX root = paper.tex
%
% Introduction
%
% Lead currently:  Kyle Mandli
%

\section{Introduction}

%!TEX root = paper.tex
%
% Development Approach
%
% Lead currently:  Aron Ahmadia's 
%

\section{Development Approach}

Clawpack's development model is largely driven by the needs of its
developer community.  The Clawpack project contains core solver
functionality, a visualization suite, a general adaptive mesh
refinement code, a specialized geophysical flow code, and a
massively parallelized Python framework.  Changes to the core solvers
and visualization suite have a downstream effect on the other codes,
and the developers largely work in an independent, asynchronous manner
across continents and time zones. 

The Clawpack team leverages the Git distributed version control system
to coordinate development, with each major project and its development
team assigned to its own repository.  The repositories are each
publicly coordinated under the Clawpack organization on GitHub, and a
top-level \texttt{clawpack} repository is responsible for hosting
build and installation tools as well as providing a synchronization
point for the other repositories.

\subsection{GitHub}

GitHub is a free provider of public Git repositories.  In addition to
repository hosting, the Clawpack team uses GitHub for issue tracking,
code review, automated continuous integration via Travis CI, and test
coverage tracking via Coveralls.  The issue tracker on GitHub
automatically recognizes cross-repository references, simplifying
communication between Clawpack developer sub-teams.  The Travis CI
services, which provides free continuous integration for publicly
developed repositories on GitHub, runs Clawpack's test suites on
proposed changes to the code base, and through a connection to
the Coveralls service, reports on any test failures as well as changes
to test coverage.

\subsection{Submodules}

As mentioned earlier, the \texttt{clawpack} repository serves as an
installation and synchronization point for the other repositories.
From a Computer Science perspective, the \texttt{clawpack} repository
holds \textit{references} to a version of the code in each
repository.  

The main Clawpack repositories are:
\begin{itemize}
    \item \texttt{clawpack} - Installation, coordination of other repositories
    \item \texttt{riemann} - Riemann solvers used by all the other projects
    \item \texttt{visclaw} - Visualization suite used by all the other projects
    \item \texttt{classic}
    \item \texttt{amrclaw}
    \item \texttt{geoclaw}
    \item \texttt{pyclaw}
\end{itemize}

Additional repositories contain documentation and extended examples of
using the code: 
\begin{itemize}
    \item \texttt{doc}
    \item \texttt{apps}
\end{itemize}

We sometimes refer to \texttt{riemann} and \texttt{visclaw} as
\textit{upstream} repositories, since their changes affect the
remaining repositories in the list, which we usually refer to as
\textit{downstream} repositories.

Typically, the Clawpack developers advance the master development
branch of the repository any time a major feature is added or a bug is
fixed in one of the other repositories.  When a feature is added to an
upstream repository that affects other repositories, the addition of
this change is coordinated with the downstream repositories by
simultaneously advancing the version of the upstream repository as
well as all affected downstream repositories.  

Git submodules are very powerful, but many developers, both new and
experienced, find them confusing and challenging to work with.
However, they have proved invaluable in allowing the Clawpack team to
work asynchronously on sub-projects while reusing and maintaining
common software infrastructure.

\subsection{Contributing}

\textbf{TODO: Revise this section, remove redundancy}

Scientist programmers are often discouraged from sharing code
due to existing reward mechanisms and the fear of being "scooped".
In fact, scientific communities that openly share and develop code
have an advantage because each researcher can leverage the work of
many others \cite{turk2013scaling}.

Over the past twenty years, a great number of users have written
additional code, extending Clawpack with new Riemann solvers,
algorithms, and domain-specific problem tools.  Most of this code
has not made it back into the core library.  With Clawpack 5.x,
we are trying to encourage contributions from a broad community, with
tools like distributed version control and open discussions on 
the mailing lists and issue trackers.
Clawpack is supported by a community of user-developers whose
collaboration

Clawpack uses the distributed version control software Git.
The main repository is hosted on GitHub.  Anyone may contribute,
for instance by developing new code, reporting bugs, or suggesting
improvements.

Bugs and feature requests are posted as issues on the tracker that
is part of the Github repository.  The tracker provides a page for
discussing the issue.

The primary development model
is typical for Github projects: a contributor forks the repository on Github,
and develops improvements in a branch that is pushed to her own fork.
She issues a "pull request" (PR) when the branch is ready to be merged
into the main repository.  Increasingly, contributors are also using
PRs as a way to conveniently post preliminary or prototype code for
discussion prior to further development.

After a PR is issued, other developers -- including one or more of the
maintainers for the corresponding repository -- reviews the code.  The Travis
CI server also automatically runs the tests on the proposed new code.  The test
results are visible on the Github page for the PR.  Usually there is some
iteration as developers suggest improvements, request more testing, etc.
Once the tests are passing and it is agreed that the code is acceptable, a
maintainer merges it.

\subsection{Releases}

\textbf{TODO: Add release information}

%!TEX root = paper.tex
%
% Advances
%
% Lead currently:  None
%

\section{Advances} \label{sec:advances}

%!TEX root = paper.tex
%
% Global Changes
%
% Lead currently:  None
%

\subsection{Global Changes}
\begin{itemize}
    \item reordering indices
    \item input variables in setrun.py (discuss Python interface to Fortran)
\end{itemize}

\todo{Describe the changes that are \clawpack wide.}
%!TEX root = paper.tex
%
% Riemann
%
% Lead currently:  David
%

\subsection{Riemann: A Community-Driven Collection of Approximate Riemann Solvers}
The methods implemented in Clawpack, and all modern Godunov-type methods for
hyperbolic PDEs, are based on the solution of Riemann problems.  A Riemann
problem is the Cauchy initial value problem posed by a first-order system of
hyperbolic PDEs and a piecewise-constant initial state with a single
discontinuity.  Riemann problems are central to hyperbolic PDEs because
discontinuities arise spontaneously in their solution.  The main theoretical
and numerical difficulties of hyperbolic problems involve the prescription of
physically correct weak solutions -- in other words, understanding the behavior
of the solution at discontinuities.  The Riemann solver is an algorithm that
encodes the specifics of the hyperbolic system to be solved, and it is the only
routine (other than problem-specific setup) that needs to be changed in order
to apply the code to different hyperbolic systems.  In some cases, the Riemann
solver may also be designed to enforce physical properties like positivity
(e.g., for the water depth in GeoClaw) or to account for forces (like that
of gravity) that may be balanced by flux terms.

For nonlinear systems, the exact solution of the Riemann problem is computationally
costly and may involve both discontinuities (shocks and contact waves) and
rarefactions.  It is almost always preferable to employ inexact Riemann solvers
that approximate the solution using discontinuities only.  Only the latter type
of Riemann solvers may be used in Clawpack.  For some systems, like the Euler
equations of compressible flow, a range of approximate solvers (giving 
more or less accurate results) has been developed in the literature.

One of the main commonalities across all packages in the Clawpack suite is the
use of a standard interface for Riemann solver routines.  This ensures that new
solvers or solver improvements developed for one package can immediately
be used by all packages.  To further facilitate this sharing and to avoid 
duplication, Riemann solvers are (with rare exceptions) not maintained under
the other packages but are collected in a single repository named Riemann.
Users who develop new solvers are strongly encouraged to submit them to the
Riemann repository.

In the Fortran-based packages (Classic, AMRClaw, and GeoClaw) the Riemann
solver is selected at compile-time by modifying a problem-specific Makefile.
In PyClaw, the Riemann solver to be used is selected at run-time.  This is
made possible by compiling all of the Riemann solvers (when PyClaw is installed)
and generating Python wrappers with f2py.  For PyClaw, Riemann also provides
useful metadata (such as the number of equations, the number of waves, and
the names of the conserved quantities) for each solver.

Whereas most existing codes for hyperbolic PDEs use Riemann solvers to
compute fluxes, Clawpack Riemann solvers instead compute the waves 
(or discontinuities) that make up the approximate Riemann solution.
Clawpack is also unusual in that it optionally makes use of transverse
Riemann solvers, responsible for computing transport between cells that
touch only along an edge or a corner.

%!TEX root = paper.tex
%
% Classic
%
% Lead currently:  rjleveque
%

\subsection{\classic}
The \texttt{classic} repository contains code implementing the wave
propagation algorithm on a single uniform grid, in much the same form as the
original \clawpack 1.0 version of 1994 but with various enhancements added
through the years.  Following the introduction of \clawpack 4.4 the
three-dimensional routines were left out of the Python user interfaces and
plotting routines.  These have been reintroduced in \clawpack 5.  Additionally
the OpenMP shared-memory parallelism capabilities have been extended to the
three-dimensional code.
%!TEX root = paper.tex
%
% AMRClaw
%
% Lead currently:  Marsha Berger
%

\subsection{\amrclaw}
The \amrclaw repository performs block structured adaptive mesh
refinement \cite{Berger:1984ui,Berger:1989gg} for both 
\clawpack and \geoclaw  applications.
A short overview is given
here to set the stage for a description of recent changes.

Currently \amrclaw includes the ability to
\begin{itemize}
\item
coordinate the flagging of points where refinement is needed
either using specified criteria, pre-specified regions, or Richardson extrapolation;
\item
organize the flagged points into efficient refined grid
patches at the next finer level;
\item
initialize newly created fine grids, both solution and
auxiliary arrays (e.g. velocity fields or bathymetry);
\item
orchestrate the time stepping allowing refinement levels to take
different time steps (called sub-cycling in time)
\item
interpolation for ghost cells on fine patches needed before a time step can be taken; and
\item
maintenance of conservation involving correction waves at
the patch boundaries between fine and coarse grids as well as
underneath fine grids.
\end{itemize}

These algorithms have been discussed in detail in
\cite{Berger:1998ia,Berger:2011du}. Roughly one third of the files in \amrclaw
have to be modified for \geoclaw specifics. For example, to maintain a level
sea-level when interpolating the sea-level itself must be interpolated rather than the depth which is the conserved quantity being kept track of.  There are 135 files at this moment 
in the two-dimensional version of \amrclaw.
Forty-five of them are replaced by a \geoclaw-specific file of the
same name but in the \geoclaw 2D branch. Recently three-dimensional version of  \amrclaw has also been brought up to parity with the two-dimensional branch which has seen the most development due to the needs of \geoclaw development.

Parallel efficiency and overall runtime is one aspect that has received more
attention in \clawpack 5. This was partly motivated by the need to run larger
simulations, such as storm surge \cite{Mandli:ws}.  The new emphasis on three-
dimensional AMR applications has also increased interest in this area.  Since
the target machine for use with \clawpack has been single workstations or nodes
of a cluster, the leveraging of multi-core technologies have been the primary
emphasis with parallelizing the software.  Other frameworks including
\forestclaw \cite{Burstedde:we} and \boxlib \todo{citation for pyboxlib?} also
exist and are being developed in parallel with \amrclaw to provide scalable
calculations but on large distributed machines.

Due to the emphasis on multi-core technologies \clawpack has been parallelized
using OpenMP directives based grid-based decomposition.  The main paradigm in
structured AMR is a loop over all grids that exist at a level, where some
operation is done on each grid (e.g. taking a time step, finding ghost cells,
conservation updates, etc.). This lends itself easily to a {\tt parallel for}
loop construct where  each iteration of the loop corresponds to a grid at that
level. Dynamic scheduling is used with a chunk size of one, so that one thread
is assigned one grid at a time.  To help with load balancing, grids at each
level are sorted from largest to smallest workload when they are first created,
using the total number of cells in the grid as an indicator of work. This same
approach is now used in three-dimensions as well.   Note that this approach
causes a memory bulge. Each thread must have its own scratch arrays to save the
incoming and outgoing waves and fluxes for future conservation fix-ups.  The
bulge is directly proportional to the number of threads executing.

%For stack-based memory allocation per thread, the use of the environmental variable {\tt OMP\_STACKSIZE} to increase the limit is necessary.

\missingfigure{Provide some parallel efficiency results.}
\todo{(Kyle) I have data for this and can add a plot if need be.}

One of the other significant additions to \amrclaw algorithmically was the
ability to copy meta-data into newly created grids from their older counterparts
if they overlapped in the correct way.  This can save effort if the computation
of the meta-data is expensive.  In \geoclaw, bathymetry is included in this
meta-data and can be costly to compute for a given cell due to multiple data
sources, mapping on the sphere and interpolation.  Since it is a common
occurrence for a new grid to overlap at least partially older grids, this
ability saves significant computational overhead associated with recomputing
these quantities.

Another new capability just added to both two and three-dimensions is spatially
varying boundary conditions.  For a single grid, it is a simple matter to
compute the location of the ghost cells that extend outside the computational
domain and set them appropriately. With AMR however, the boundary condition
routine can be called for a grid located anywhere in the domain, and may contain
fewer or larger numbers of ghost cells. The boundary condition routines must be
written in a rather unusual way that does not assume it is always setting the
same number of ghost cells, or that the same number of reflected cells inside
the domain always exist. A new computational example of a vortex flowing in from
one side of the boundary is now part of the {\tt Examples} directory to test
this.\todo{Confirm addition of vortex movement, I actually think this may have
been added to the test suite but not the examples.}



%!TEX root = paper.tex
%
% GeoClaw
%
% Lead currently:  Randy LeVeque
%

\subsection{\geoclaw} \textbf{Randy}
\begin{itemize}
    \item NTHMP benchmarks
    \item setaux copying for faster topography integration, recursive 
    \item new topotools and dtopotools
    \item multiple dtopo files
    \item fixed grid monitoring of max values
    \item Spatially varying boundary conditions
\end{itemize}
%!TEX root = paper.tex
%
% PyClaw
%
% Lead currently:  David Ketcheson
%

\subsection{\pyclaw}

Later 4.x releases included a number of Python-based tools for handling Clawpack input and output.  The 5.0 release includes a full-fledged Python solver in which the higher-level parts of Clawpack have been reimplemented in Python.  This new solver also includes access to the high-order algorithms introduced in SharpClaw and can be used on large distributed-memory parallel machines.  Lower-level code (whatever gets executed repeatedly and needs to be fast) from the earlier Fortran Classic and SharpClaw codes is automatically wrapped at install time using f2py.

\begin{itemize}
    \item overall structure/languages figure
    \item sharpclaw
    \item petclaw
    \item f2py
    \item pip install
    \item IPython notebooks
\end{itemize}
%!TEX root = paper.tex
%
% VisClaw
%
% Lead currently:  None
%

\subsection{visclaw}
\begin{itemize}
    \item Iplotclaw
    \item creation of html pages
    \item JSAnimation
    \item (Once 3d is working, write a separate paper on visclaw as a general tool?)
\end{itemize}

%!TEX root = paper.tex
%
% Future Work
%
% Lead currently:  None
%

\section{Conclusions} \label{sec:conclusions}

\clawpack has evolved over the past 20 years from its genesis as a small and
focused software package that two core developers could manage without
version control.  It is now an ecosystem of related projects that share a core
philosophy and some common code (notably Riemann solvers and visualization
tools), but that are aimed at different user
communities and that are developed by overlapping but somewhat distinct
groups of developers scattered at many institutions.  The adoption of better
software engineering practices, in particular the use of Git and GitHub as an
open development platform and the use of pull requests to discuss proposed
changes, has been instrumental in facilitating the development of many of the
new capabilities summarized in this paper.  

\subsection{Future Plans} \label{sub:future}

The \clawpack development team continues to look forward to new ideas and
efforts that will allow great accessibility to the project as well as new
capabilities that the core development team has not thought of.  To this end a
number of the broad efforts that are being considered for the next major release
of \clawpack include
\begin{itemize}
    \item An increased librarization effort with the Fortran based sub-packages,
    \item An extensible and more accessible interface to the Riemann solvers,
    \item An effort to allow \pyclaw and the \clawpack Fortran packages to rely
    on more of the same code-base,
    \item An increased emphasis on a larger development community,
    \item More support for new frameworks such as \forestclaw,
    \item A refactoring of the visualization tools in \visclaw, along with
    support for additional backends, particularly for three-dimensional results
    (e.g.
\texttt{VisIt}\footnote{\url{https://visit.llnl.gov}}, 
\texttt{ParaView}\footnote{\url{http://www.paraview.org/}}, or 
\texttt{yt}\footnote{\url{http://yt-project.org/}}).
\end{itemize}


%!TEX root = paper.tex
%
%


\section*{Acknowledgments} 

We wish to thank the many people who have made contributions to the 
\clawpack software over the years, including users who have submitted bug
reports (or even better, bug fixes!) or have suggested
improvements to the software.  We also acknowledge many funding sources that
have contributed to the developments outlined in this paper, including
NSF grants DMS-1216732, DMS-1419108, and EAR-1331412,
DOE Office of Advanced Scientific Computing grant DE-FG02-88ER25053,
KAUST OCRF grant 2156 CRG3,
the University of Washington Department of Applied Mathematics,
\alert{others?}




\bibliographystyle{siamplain}
\bibliography{paper}

\end{document}
